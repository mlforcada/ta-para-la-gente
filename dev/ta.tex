\documentclass{beamer}
%\documentclass[handout]{beamer}
 \mode<presentation>{ 
 \usetheme{Antibes}%\usetheme{Warsaw}  
\setbeamercovered{transparent}
   %\usecolortheme{albatross} % for inverted color
% \usecolortheme{wolverine} % way too flashy!
    %\usefonttheme{structurebold}
    %\usetheme{Berkeley}
    % or ...    \setbeamercovered{transparent}
    % or whatever (possibly just delete it)
 
 } 
%\mode<handout>

 \frametitle{Traducció automàtica estadística}


\usepackage[english]{babel}
\usepackage[utf8]{inputenc}
\usepackage{times}
\usepackage{url}
\usepackage[T1]{fontenc}
\usepackage{alltt}
\usepackage[normalem]{ulem}

\newcommand{\Pair}[2]{\texttt{#1}\(\leftrightarrow\)\texttt{#2}}
\newcommand{\pair}[2]{\texttt{#1}\(\to\)\texttt{#2}}

\title[Traducció automàtica]{Traducció automàtica}

\author[M.L.\ Forcada]{Mikel L.\ Forcada\inst{1,2}}

\institute[Universitat d'Alacant i Prompsit]{ 
\inst{1}Departament de Llenguatges i Sistemes Informàtics,\\
Universitat d'Alacant,  03071 Alacant \\[0.2cm]
\inst{2}Prompsit Language Engineering, S.L., \\ Edifici Quorum III, Av. Universitat s/n, 03202 Elx}


\date[06/07/2011]{La linguística i les seues aplicacions en la societat\\ Castelló, 6 de juliol de 2011}


\newcommand{\tu}[2]{(\texttt{``#1''},\texttt{``#2''})}

\newcommand{\catala}{català}
\newcommand{\catalana}{catalana}
\newcommand{\espanyol}{espanyol}
\newcommand{\delespanyol}{de l'espanyol}
\newcommand{\lespanyola}{l'espanyola}
\newcommand{\menjo}{menjo}
\newcommand{\uneixi}{uneixi}
\newcommand{\atura}{atura}
\newcommand{\empha}[1]{\emph{#1}\/}

%\newcommand{\curs}{Traducció Automàtica: Fonaments i Aplicacions}
%\newcommand{\llocidata}{Universitat d'Alacant, 2004}

\AtBeginSection[]
{
  \begin{frame}<beamer>{Outline}
    \tableofcontents[currentsection,currentsubsection]
  \end{frame}
}


\begin{document}

\frame{\maketitle}
% \begin{frame}
% \maketitle
% \titlepage
% \end{frame}

\begin{frame}<beamer>
\frametitle{Contents}
\tableofcontents
\end{frame}


%%%%%%%%%%%%%%%%%%%
\section{Què és la traducció automàtica (TA)?}
\begin{frame}
\frametitle{ Què és la traducció automàtica (TA)? /1}

{
{La \empha{traducció}, \ldots }
\pause

{\ldots mitjançant un \empha{sistema informàtic} \ldots}
\pause

{\ldots (ordinador(s) +
  programes) \ldots}
\pause

{\ldots de \empha{textos informatitzats} en la \empha{llengua
    origen} (LO)\ldots}
\pause

{\ldots a \empha{textos informatitzats} en la \empha{llengua meta}
  (LM).}

}
\end{frame}

%%%%%%%%%%%%%%%%%%%

\begin{frame}
\frametitle{ Què és la traducció automàtica (TA)? /2}

Esquemàticament:

\begin{center}
\begin{tabular}{ccccc}
\textbf{Text LO} & $\to$ & \framebox{\parbox{3.5cm}{Sistema de traducció
    automàtica}} & $\to$ & \parbox{2.0cm}{\textbf{Text LM} (en brut)} \\ 
\end{tabular}
\end{center}

\end{frame}
\begin{frame}
  \frametitle{Què \textbf{no} és?}

  Hi ha altres tecnologies de traducció que no són traducció
  automàtica. 

  Destaquen els sistemes de \textbf{traducció assistida per ordinador} basats en \textbf{memòries de traducció}.

  En parlarem una miqueta al final de la lliçó si tenim temps!.
\end{frame}

%%%%%%%%%%%%%%%%%%%

\section{Aplicacions de la TA}
\begin{frame}
\frametitle{ Aplicacions de la TA /1}


Dos grans grups:
\begin{itemize} 

\item\empha{Assimilació:} traducció \textbf{efímera},
    idealment \textbf{instantània}, per a la revisió o la
    \textbf{comprensió} de documents en una altra llengua que no
    coneixem. P.ex., \empha{``navegació traduïda'' per internet}, \empha{xat
      (\textit{chat}) multilingüe}, etc.

  \item\empha{Disseminació:} traducció \textbf{permanent}, idealment
    \textbf{amb pocs errors}, per a la \textbf{publicació}. P.ex.,
    producció d'\textbf{esborranys} per a corregir (\textit{posteditar}).
\end{itemize}

\end{frame}
%%%%%%%%%%%%%%%%%%%
\begin{frame}
  \frametitle{Assimilació: un text en irlandés/1}

  \empha{Is ar an Oileán Fada a bhí mé féin agus m’fhear céile ag fanacht
  nuair a rinne muid an turas sin. Dúradh linn fanacht cois farraige
  ag am díthrá agus thuirling an muireitleán anuas chun sinn a
  thabhairt ar bord. Bhí dream beag ar an turas mar níor thóg an
  t-eitleán sin thar deichniúr in am ar bith. Níl cur síos ar na
  radhairc a chonaic muid ar an aistear go dtí an mhórsceir.  Taobh
  istigh d’uair an chloig, bhí an mhórsceir féin bainte amach againn
  agus ba dhochreidte na radhairc a chonaic muid os ár gcomhair.}
\begin{itemize}
\item Qui s'anima a dir de què va?

\end{itemize}
  \end{frame}

%%%%%%%%%%%%%%%%%%%
\begin{frame}
  \frametitle{Assimilació: traducció de Google Translate}

  \empha{És a Long Island i estava esperant al meu marit quan vam fer el
  viatge. Ens van dir de quedar a la vora del mar a \textbf{díthrá} i va
  aterrar passat el vaixell que ens portarà a bord. Petit grup va ser
  en el viatge no havia van prendre l'avió més de deu en poc temps. No
  hi ha una descripció de les escenes que vam veure en el viatge cap a
  l'escull. En qüestió d'hores, l'escull que s'han llaurat va ser
  increïble i les escenes que vam veure davant nostre.}
\begin{itemize}
\item El sistema no reconeix el mot \empha{díthrá}
\end{itemize}

\end{frame}
%%%%%%%%%%%%%%%%%%%
\begin{frame}
  \frametitle{Assimilació: el que volia dir (poc més o menys)}

\empha{Era a Long Island on m'estava amb al meu marit quan vam fer el
  viatge. Ens van dir de quedar a la vora del mar a l'hora de la
  \textbf{baixamar}, on va aterrar l'hidroavió que ens portaria a bord. Era un
  petit grup, el del viatge, perquè no prenien aquell avió més de deu
  persones en cap moment. No hi ha com descriure les escenes que vam
  veure en el viatge cap a l'escull. En qüestió d'hores, havíem
  abastat l'escull i van ser increïbles les vistes que vam veure
  davant nostre.}

\begin{itemize}
\item \empha{dithrà} = \empha{baixamar}
\end{itemize}

\end{frame}


%%%%%%%%%%%%%%%%%%%

\begin{frame}
\frametitle{ Aplicacions de la TA /2}

\empha{Postedició i preedició}: els professionals col·laboren amb el
  sistema de TA en aplicacions de \textbf{disseminació}:
\begin{itemize}
\item\empha{Postedició:} correcció del text traduït en brut
  per a fer-lo adequat al propòsit previst (de seguida veurem un exemple).
\item Un pas més: \empha{preedició} o preparació del text per a evitar
  lèxic o construccions que donen problemes de traducció coneguts amb un
  sistema de traducció automàtica. 
  \begin{itemize}
  \item Interessant quan el mateix text original s'ha de traduir a moltes llengues meta (una preedició pot estalviar moltes postedicions).

  \end{itemize}
\end{itemize}

  Conéixer bé \empha{com funciona} el sistema de TA ajuda molt en
  ambdues tasques.

\end{frame}
%%%%%%%%%%%%%%%%%%%


\begin{frame}
\frametitle{ Aplicacions de la TA /3}

{ {Alternativa en cas de preedició repetitiva: \empha{llenguatge
      controlat}.}
\begin{itemize}\setlength{\itemsep}{0pt}
{\item Els autors escriuen ja pensant en el tractament automatitzat
  del text.}
{\item S'eviten lèxic i construccions problemàtiques per al sistema de TA per minimitzar (no \empha{eliminar}) la postedició.
  \begin{itemize}
  \item (en) \empha{replace} \(\to\) (ca) \empha{substituir} (en: \empha{substitute}) o \empha{tornar a col·locar} (en: \empha{put back})?
  \end{itemize}

}
{\item Consistència d'estil, comprensibilitat, mantenibilitat dels documents originals.}
{\item Però els autors han de conéixer i aplicar el llenguatge controlat!}
{\item Se'ls pot ajudar amb eines informàtiques.}
\end{itemize}
%\textbf{[transparències introducció progressiva?]}
}
\end{frame}

%%%%%%%%%%%%%%%%%%%

\begin{frame}
  \frametitle{Exemple de postedició: text original en portugués} 

  \empha{Nova Iorque teve ontem uma noite de festa, após a aprovação
    durante a tarde (madrugada em Lisboa) pelo Senado estadual da lei
    que reconhece o direito ao casamento homossexual, por 33 votos
    contra 29, após anos de falhanço nesta câmara-alta.  O \textbf{projecto} de
    lei tinha até aqui sido aprovado quatro vezes pela Assembleia do
    Estado, mas tinha sido sempre \textbf{rejeitada} pelo Senado de Nova
    Iorque, que agora se torna no sexto \textbf{estados} dos EUA a permitir o
    casamento entre pessoas do mesmo sexo.}

  \begin{itemize}
  \item El text original te dues errades (en negreta).
  \end{itemize}

\end{frame}

\begin{frame}
  \frametitle{Traducció en brut al valencià (apertium-pt-ca)}

  \empha{Nova York va tenir ahir una nit de festa, *após l'aprovació
    durant la tarda (matinada a Lisboa) pel *Senado *estadual de la
    llei que reconeix el dret a les noces homosexuals, per 33 vots
    contra 29, *após anys de fracàs en aquesta càmera-alta.  El
    projecte de llei tenia fins a aquí estat aprovat quatre vegades
    per l'Assemblea de l'Estat, però havia estat sempre rebutjada pel
    *Senado de Nova York, que ara es fa en el sisè estats dels *EUA a
    permetre les noces entre persones del mateix sexe.}

  \begin{itemize}
  \item Els mots desconeguts pel sistema estan marcats amb un asterisc.
  \end{itemize}
\end{frame}

\begin{frame}
  \frametitle{Text posteditat (postedició ràpida)}

  \empha{Nova York va \textbf{passar} ahir una nit de festa,
    \textbf{després de} l'aprovació durant la \textbf{vesprada}
    (matinada a \textbf{Castelló}) pel Senat \textbf{d'aquest estat}
    de la llei que reconeix el dret \textbf{al matrimoni dels}
    homosexuals, per 33 vots contra 29, \textbf{després d'}anys de
    fracàs en aquesta \textbf{cambra} alta.  El projecte de llei
    \textbf{havia} estat \textbf{fins ara} aprovat quatre vegades per
    l'Assemblea de l'Estat, però havia estat sempre \textbf{rebutjat}
    pel \textbf{Senat} de Nova York, que ara es \textbf{converteix} en
    el sisè \textbf{estat} dels \textbf{EUA} a permetre \textbf{el
      matrimoni} entre persones del mateix sexe.}

  \begin{itemize}
  \item Els canvis s'indiquen amb negretes.
  \end{itemize}
\end{frame}
%%%%%%%%%%%%%%%%%%%


\begin{frame}
\frametitle{ Aplicacions de la TA /4}

{
{En primera aproximació, la postedició és convenient quan 
$$\textbf{cost}\left(\mbox{\begin{tabular}{c}traducció automàtica\\ +\\
  postedició professional\end{tabular}}\right) < \textbf{cost}(\mbox{traducció professional}).  
$$
(la fórmula amaga molts detalls del \empha{cost}).
}

{Perquè la postedició siga eficient:}
\begin{itemize}\setlength{\itemsep}{0pt}
{\item no pot afectar a una gran part del text (máx. 20\%--30\%, si no, s'abandona i es comença des de zero)}
{\item cal ser competent en la llengua meta $\to$ generar un
  text genuí a partir del text en brut}
{\item s'ha de tenir en compte el propòsit del text traduït}
{\item cal conéixer el sistema de TA $\to$ reconéixer l'origen dels
  errors, predir-ne el comportament}
\end{itemize}
}
\end{frame}


\section{Com funciona la TA?}

\begin{frame}
  \frametitle{Com funciona la TA? /1}
Dues grans tecnologies: 
\begin{itemize}
\item la \textbf{traducció automàtica basada en regles (TABR)} i
\item la \textbf{traducció automàtica basada en corpus (TABC)} 
\end{itemize}
Per descomptat, hi ha aproximacions \empha{híbrides}.
\end{frame}
%%%%%%

\begin{frame}
  \frametitle{Com funciona la TA? /2}

La \textbf{traducció automàtica basada en regles (TABR)}
  \begin{itemize}
  \item aproximació dominant en productes comercials fins fa poc
  \item experts compilen diccionaris, regles de traducció i escriuen programes que els apliquen
  \item parlaré fonamentalment d'aquest tipus de TA
  \item exemples: \empha{Systran}, \empha{Lucy}, \empha{Apertium}, \empha{Reverso}, etc.
  \end{itemize}
\end{frame}


\begin{frame}
  \frametitle{Com funciona la TA? /3}
La \textbf{traducció automàtica basada en corpus (TABC)} (i, molt especialment, la \empha{traducció automàtica estadística}:
  \begin{itemize}
  \item és ara l'aproximació més puixant (l'''estat de la qüestió'')
  \item els sistemes ``aprenen a traduir'' a partir d'enormes corpus de text bilingüe alineat oració a oració
  \item el que aprenen són complexos models estadístics (diccionaris probabilístics de mots i segments més llargs, probabilitats de reordenaments, etc.)
  \item exemple: \empha{Google Translate}
  \item en parlarem una miqueta només, més avant.
  \end{itemize}
\end{frame}

\begin{frame}
\frametitle{ Com funciona la TA? /4}

\textbf{Primera aproximació} [!!]: Traduir textos és
\empha{traduir oracions}.

Traduir oracions suposa:

{
\begin{itemize}
  {\item \empha{Construir una interpretació} (un
    \empha{significat}) a partir de l'oració en LO.}  {\item
  \empha{Construir una oració en LM} a partir de la interpretació.}
\end{itemize}
}
\end{frame}

\begin{frame}
\frametitle{ Com funciona la TA? /5}

\empha{Principi de composicionalitat [semàntica]}:


{
{La \textbf{interpretació d'una oració} es construeix \ldots}

\pause

{\ldots a partir de les \textbf{interpretacions dels
    mots} \ldots}

\pause
{\begin{center}\textit{Escriuen \empha{cartes}} $\neq$ \textit{Escriuen \empha{articles}}\end{center}}

\pause
{\ldots \empha{component-les} seguint les agrupacions indicades 
per l'\textbf{estructura sintàctica} de
  l'oració.}
\pause
{\begin{center} \textit{Israel amenaça Palestina} $\neq$ \textit{Palestina amenaça Israel} \end{center}}
\pause
Això permet que puguem assignar interpretació a oracions que no hem sentit mai:

\pause
{\begin{center} \textit{Arranqueu el projector del sostre i llanceu-lo per la finestra!} \end{center}}

}

\end{frame}

%%%%%%%%%%%%%%%%%%%

\begin{frame}
\frametitle{Com funciona la TA? /6}

Però alerta!  Les \textbf{oracions} poden ser \textbf{ambigües} (\'es
a dir, tenir m\'es d'una interpretaci\'o):

\begin{itemize}
\item perqu\`e els mots tenen m\'es d'una interpretaci\'{o}
  (ambig\"uitat \empha{l\`exica})
 

\item perqu\`{e} l'oraci\'{o} t\'{e} m\'{e}s d'una possible an\`alisi
  sint\`actica (ambig\"uitat \empha{sint\`actica})

\item per ambdues coses alhora.
\end{itemize}
(en veurem exemples més endavant)

Elegir la interpretaci\'{o} correcta no \'es trivial per a un sistema
  inform\`atic (normalment només pot usar part del \textbf{cotext}).

%\textbf{[posar exemples ací?]}

\end{frame} 



%%%%%%%%%%%%%%%%%%%

\begin{frame}
\frametitle{ Com funciona la TA? /7}

Esquemàticament:

{
{
\begin{center}
\begin{tabular}{cccccc}
\parbox{1.8cm}{Oració LO}
& $\to$ 
& interpretació 
& $\to$ 
& \parbox{1.8cm}{Oració LM} \\ 
\end{tabular}
\end{center}
}

{ En alguns sistemes de TA s'intenta representar directament les
  \empha{interpretacions} amb una \empha{interlingua} (un llenguatge estructurat
  artificial).}  }
\end{frame}

%%%%%%%%%%%%%%%%%%%%%%%%%%%%%%%%%%%
\begin{frame}
\frametitle{ Com funciona la TA? /8}

 { {Però... els \textbf{traductors
      professionals} realment necessiten \textbf{interpretar} o
    comprendre \textbf{completament} una oració per a
    traduir-la?}

{
\begin{center}
  \textsl{``... interacciones independientes del espín en unidades de la
  sección eficaz del neutrino de Dirac...''} $\to$ \\
  \textsl{\empha{``... interaccions independents de l'espín en unitats de la secció
  eficaç del neutrí de Dirac...''} (\textbf{traductora i física de partícules?})}
\end{center}
}
{
\begin{center}
\textsl{``\ldots tornillos que unen el volante de inercia al árbol de levas}
 $\to$ \\
\textsl{\empha{``\ldots caragols que uneixen el volant d'inèrcia a l'arbre
de lleves \ldots''} (\textbf{traductora i enginyera mecànica?})}
\end{center}
} {No: \textbf{Transformen estructures} o patrons i
\textbf{substitueixen el lèxic} (parant especial esment al
terminològic).} 
}

\end{frame}

%%%%%%%%%%%%%%%%%%%

\begin{frame}
\frametitle{ Com funciona la TA?  /9}

{
{Això permet fer la \textbf{segona aproximació} [!!], també anomenada \empha{aproximació de transferència}}
\begin{itemize}\setlength{\itemsep}{0pt}
{\item[] La majoria dels sistemes de TA \textbf{no construeixen
    completament la interpretació}, \ldots}

\pause
{\item[]\ldots sinó que \textbf{transformen l'estructura sintàctica} de
  l'oració en LO en una estructura sintàctica vàlida per a l'oració en
  LM i\ldots}

\pause
{\item[]\ldots \textbf{substitueixen} els  mots de l'oració en LO per
  equivalents adequats en LM\ldots }

\pause
{\item[]\ldots fent les dues operacions \textbf{bastant
    independentment}.}
\end{itemize}
}

\end{frame}

\begin{frame}
  \frametitle{Com funciona la TA/10}
Esquemàticament:
\begin{small}
\begin{center}
\parbox{1.0cm}{Oració LO} $\to$
\framebox{\parbox{0.8cm}{Anà\-lisi}} $\to$ 
\parbox{1.0cm}{repre\-senta\-ció abs\-tracta LO} $\to$
\framebox{\parbox{1.2cm}{Transfe\-rència}} $\to$
\parbox{1.0cm}{repre\-senta\-ció abs\-tracta LM} $\to$
\framebox{\parbox{0.8cm}{Gene\-ració}} $\to$
\parbox{1.0cm}{Oració LM} 
\end{center}
  \end{small}

\end{frame}

\begin{frame}
  \frametitle{Com funciona la TA/11}

  La tercera aproximació: arquitectura de transformació
  (\empha{transformer}):

  \begin{itemize}
  \item Per a traduir entre llengües similars, no cal fer l'anàlisi sintàctica completa de l'oració.
  \item Només es fa \empha{on cal}: anàlisi \empha{parcial} (la resta, mot a mot)
    \begin{itemize}
    \item Anàlisi morfològica: \empha{dit} \(\to\) \{\texttt{\textbf{dir}.verb.part.m.sg}, \texttt{\textbf{dit}.n.m.sg}\}
    \item Desambiguació categorial: se n'elegeix una (\empha{has dit} \(\not\to\) \empha{*has dedo})
    \item Identificació de patrons locals on caldrà establir la
      concordança local, fer petits reordenaments (p.ex. \textbf{n--adj}: \empha{coixí groc} \(\not\to\) \empha{*almohada amarillo})
    \end{itemize}
  \item Aquesta aproximació s'usa més que no us penseu!
  \end{itemize}

\end{frame}

%%%%%%%%%%%%%%%%%%%%%%%%%%%%%%%%%%%%%%%%%%%
\begin{frame}
\frametitle{ Com funciona la TA? /12}
{
  
  { Per a programar un sistema de TA (basat en regles) cal \textbf{formular
      tots els processos de traducció de forma} \empha{\textbf{explícita}} i
    \empha{\textbf{mecanitzable}}} {(adéu ``intuïció
    lingüística''!).}
  
  {A més, la mecanització ha de ser \empha{eficient} (programes
    \textbf{ràpids} i \textbf{compactes}) i s'ha de
    \textbf{dur a terme en un temps raonable}:}
\begin{itemize}\setlength{\itemsep}{0pt}
{\item Això exigeix una \textbf{reflexió lingüística}
  (traductològica) sobre els processos de traducció per part dels
  dissenyadors del sistema.}
{\item A més, pot comportar més \textbf{aproximacions},
  \textbf{simplificacions}, \textbf{compromisos} i
  \textbf{sacrificis}.}
\end{itemize}
}
\end{frame}

%\section{Traducció automàtica estadística}
\begin{frame}
 \frametitle{Com funciona la TA/13}

\textbf{Traducció automàtica basada en corpus}
 \begin{itemize}
 \item Fins ara hem parlat de \empha{traducció automàtica basada en regles} i diccionaries escrits per experts.
\item Contràriament, la \textbf{traducció automàtica basada en corpus} es basa en la idea que ``les traduccions existents contenen més solucions a més problemes de traducció que qualsevol altre recurs disponible'' (Pierre Isabelle)
\item Entre la traducció automàtica basada en corpus, destaca la
\textbf{traducció automàtica estadística} per la seua hegemonia actual.
 \end{itemize}


\end{frame}


\begin{frame}
 \frametitle{Com funciona la TA? /14}

\textbf{Entrenament d'un sistema estadístic:}

En \empha{traducció automàtica estadística}, s'\empha{entrena} el
sistema a partir de
\begin{itemize}
\item un \empha{gran corpus de text bilingüe alineat oració a
  oració}  (o siga, un corpus bilingüe d'oracions)
\item  i d'un \empha{corpus de text monolingüe en la llengua meta encara més gran}. 
\end{itemize}

\end{frame}

\begin{frame}
 \frametitle{Com funciona la TA? /15}

El resultat de l'\empha{entrenament}:
\begin{itemize}
\item Diccionaris \emph{probabilístics} de mots i segments de més d'un mot (on s'associen probabilitats a unitats de traducció com ara \tu{con cargo a}{a càrrec de} (probable) o \tu{con cargo a}{de bestreta} (improbable).
\item Models que assignen probabilitats a seqüències de paraules en la llengua meta (de manera que \texttt{``dues de les tres són''} siga més probable que \texttt{``de són tres dues les''}.
\item Models probabilístics de reordenament (p.ex., per a obtenir ``\texttt{el cotxe\(^2\) del qual\(^1\)}'' a partir de ``\texttt{cuyo\(^1\) coche\(^2\)}'')
\end{itemize}

\end{frame}
\begin{frame}
 \frametitle{Com funciona la TA? /16}

\textbf{El procés de traducció estadística:}

\begin{itemize}
\item Aquestes probabilitats es poden veure com \empha{puntuacions
    parcials} que es combinen per a cada oració i es ponderen amb
  \empha{pesos} usant un \empha{barem} per a donar una
  \empha{puntuació global} a totes les traduccions possibles d'una
  oració donada.
\item La traducció serà la que millor \empha{puntuació global} obtinga.
\item Evidentment, no es baremen \empha{totes} les traduccions possibles (moltíssimes!): s'usen \emph{aproximacions} per a buscar només entre les que \emph{a priori} són més
probables.
\item Es tracta d'un procés computacionalment molt intensiu: per això només tenim traducció estadística aplicable des de fa poc més d'un decenni.
\end{itemize}

\end{frame}

\begin{frame}
\frametitle{Com funciona la TA? /17}


\textbf{Ajust} (o \empha{tuning}) del sistema estadístic: 

D'acord, però, quin és el \empha{pes} que cal donar en el \emph{barem} a cada
\empha{puntuació parcial}?

Ens reservem una part del \empha{corpus bilingüe} per a l'ajust:
\begin{itemize}
\item Traduïm cada oració a l'altra llengua usant diferents valors dels \empha{pesos}
\item Comparem quantitativament la \empha{semblança} entre la traducció
  automàtica i la que hi ha al corpus.
  \begin{itemize}
  \item Normalment bastant burda: es puntuen les coincidències de grups d'un mot, de dos mots, de tres mots, etc (un altre barem!).
  \end{itemize}
\item Ajustem els pesos perquè aquesta \empha{semblança} siga com més
  estreta millor per a tot el corpus d'ajust.
\end{itemize}
\end{frame}

\begin{frame}
  \frametitle{Com funciona la TA? /18}
Com a conseqüència, en traducció automàtica estadística:
\begin{itemize}
\item No calen experts per a construir el sistema.
\item És impossible puntuar totes les possibles traduccions d'una
  oració donada (ens podem perdre la millor perquè la recerca és aproximada).
\item Un sistema de traducció automàtica estadística entrenat sobre un determinat corpus produeix traduccions \empha{similars} a les del corpus.
\item La \empha{semblança} vindrà determinada per la funció de \empha{semblança} usada per a l'ajust.
\item Traduccions molt \empha{naturals} (gràcies al model
  probabilístic de la llengua meta) poden ser \empha{infidels}.
\item Canviar un mot origen pot donar lloc a una oració meta completament diferent.
\end{itemize}
\end{frame}


%%%%%%%%%%%%%%%%%%%%%%%%%%%%%%%%%%%%%%%%%%%%%

\begin{frame}
  \frametitle{ Com funciona la TA? /19} 
  
  \textbf{Per tant, tant si usem regles com estadística...}

  \begin{itemize}
  \item \empha{Podem esperar}, en el millor cas, que un bon sistema de TA 
    \begin{itemize}
    \item ens allibere de la
      part més \empha{mecànica} (mecanitzable) de la tasca de traducció i
    \item ens permeta concentrar-nos en la part més \empha{creativa} (per exemple, durant la postedició).
    \end{itemize}


  \item Però \empha{no hem d'esperar} ---per bo que siga--- que 
    \begin{itemize}
    \item comprenga
      el text, 
    \item resolga les ambigüitats sempre correctament i 
    \item produïsca
      textos en una variant genuïna de la llengua meta.
    \end{itemize}
  \end{itemize}

\end{frame}



%%%%%%%%%%%%%%%%%%%%%%%%%%%%%%%%%%%%%%%%%%%%%
\section{Per què és difícil la TA?}
\begin{frame}
  \frametitle{ Per què és difícil la TA? /1} 

   {
  {Els quatre problemes de la traducció automàtica (Arnold 2003):}
  \begin{enumerate}
   {\item El problema de l'anàlisi}
   {\item El problema de la síntesi}
   {\item El problema de la transferència}
   {\item El problema de la descripció}
  \end{enumerate}
  }

  
\end{frame}

\begin{frame}
  \frametitle{ Per què és difícil la TA? /2} 
  
  \textbf{El problema de l'anàlisi:} La forma no determina completament el
  contingut (la \empha{interpretació}). També s'anomena \empha{ambigüitat}. 
  Com hem vist, l'ambigüitat pot ser:
  \begin{itemize}
  \item \textbf{Estructural} o sintàctica:
  \begin{itemize}
  \item \empha{Portaven notícies de Grècia} (tema o procedència?)
  \item \empha{Ha venut les taronges que ha
  comprat a Joan} (Joan ven o compra?)
  \item \empha{La direcció s'oposa a implantar mesures per evitar la crisi} (realment vol evitar la crisi?)
  \end{itemize}
  \item \textbf{Lèxica} 
    \begin{itemize}
  \item \empha{Treballa en l'estudi que li han encarregat} (prepara un document
  o dissenya un taller d'artista?)
  \item \empha{El gat és sota el cotxe} (felí o màquina d'alçar cotxes?)
    \end{itemize}
  \item O fins i tot \textbf{mixta}:
    \begin{itemize}
    \item Les gallines han destrossat el sembrat però no les mates (destrucció parcial? càstig severíssim?)
    \end{itemize}
  \end{itemize}
\end{frame}


%%%%%%%%%%%%%%%%%%%%%%%%%%%%%%%%%%%%%%%%%%%%%%%%%%%%%%%%%%%%%%%%%%%%%
\begin{frame}
  \frametitle{ Per què és difícil la TA? /3} 
  
  \textbf{El problema de la síntesi} (o de la generació): El contingut
  no determina completament la forma (hi ha més d'una manera de dir el
  mateix en qualsevol llengua, però només una és \empha{genuïna} o la
  més adequada per al propòsit previst):
  \begin{itemize}
  \item \empha{Quina hora és?}
  \item \empha{Com és de tard?} (\texttt{de}: \empha{Wie spät ist es?})
  \item \empha{Quines hores són?} (\texttt{pt}: \empha{Que horas s\~{a}o?})
  \item \empha{Quantes del rellotge són?} (\texttt{de}: \empha{Wieviel Uhr ist es?})
  \end{itemize}

  Els \empha{expedients} s'\empha{obrin} o s'\empha{inicien}?

  Les sessions es \empha{clouen}, es \empha{tanquen}, es \empha{rematen} o \empha{s'alcen}?

\end{frame}

%%%%%%%%%%%%%%%%%%%%%%%%%%%%%%%%%%%%%%%%%%%%%%%%%%%%%%%%%%%%%%%%%%%%%
\begin{frame}
  \frametitle{ Per què és difícil la TA? /4} 
  
  \textbf{El problema de la transferència:} Les llengües divergeixen. És a
  dir, hi ha diferències irreductibles en la manera en que el mateix contingut
  s'expressa en llengües diferents:
  \begin{itemize}\itemsep 0ex
  \item \texttt{ca:} \empha{M'agrada nadar} (\empha{M'} objecte, \empha{agrada},
    verb, \empha{nadar} subjecte)
  \item \texttt{en:} \empha{I like swimming} (\empha{I} subjecte, \empha{like}
  verb, 
  \empha{swimming} objecte)
  \item \texttt{de:} \empha{Ich schwimme gern} (\empha{Ich} subjecte,
  \empha{schwimme}, verb,  \empha{gern}, adverbi)
  \end{itemize}
  Totes volen dir \texttt{produir\_plaer(agent=nadar(agent=jo),destinatari=jo)}
\end{frame}


%%%%%%%%%%%%%%%%%%%%%%%%%%%%%%%%%%%%%%%%%%%%%%%%%%%%%%%%%%%%%%%%%%%%%
\begin{frame}
  \frametitle{ Per què és difícil la TA? /5} 
  
  \textbf{El problema de la descripció} (represa): construir un sistema de
  traducció automàtica comporta la gestió d'una gran quantitat de coneixement,
  que s'ha d'elicitar, arreplegar, descriure, i representar en una forma útil i
  computable:
  \begin{itemize}
  \item que el femení d'\empha{orfe} és \empha{òrfena} i el plural de \empha{llapis} és \empha{llapis}
  \item que la traducció de \empha{coixí} (masc.) és \empha{almohada} (fem.)
  \item que la forma que pren \empha{en}  davant de topònim és \empha{a}
  \item que \empha{Dénia}, \empha{Olot}\ldots són topònims (i \empha{Toni} no)
  \item que cal llevar la preposició \empha{de} davant \empha{que} en català
  \item que en una seqüència \textbf{det--subst--adj}, si el substantiu canvia de gènere en traduir, cal canviar el del determinant i l'adjectiu però només si l'adjectiu concordava ja amb el substantiu\ldots
  \end{itemize}
Els sistemes estadístics tenen altres dificultats (arreplegar corpus representatius, \emph{netejar-los}, alinear-los\ldots).

\end{frame}

%%%%%%%%%%%%%%%%%%%%%%%%%%%%%%%%%%%%%%%%%%%%%%%%%%%%%%%%%%%%%%%%%%%%%


\section{Avaluació de la TA}

\begin{frame}
  \frametitle{Avaluació de la traducció automàtica /1}

  \begin{itemize}
  \item   L'avaluació de la TA és encara un tema obert i controvertit per als
  especialistes:
  \begin{itemize}
  \item Com que l'avaluació humana és molt costosa (com veurem) hi ha
     mesures d'avaluació automàtica que comparen l'eixida
    del sistema amb un o més textos de referència.
  \item La capacitat d'aquestes mesures per a predir la utilitat de la
    TA en aplicacions concretes és encara molt limitada.
  \end{itemize}
  \item   Al mateix temps, és un tema sobre el qual molta gent inexperta
  s'aventura a donar opinions, normalment excessivament optimistes o
  excessivament pessimistes.
  \item   En general se sol oblidar que cal avaluar pensant en un \textbf{propòsit}.
  \item   Imaginem-nos una aplicació concreta i veurem per què no és gens fàcil.
  \end{itemize}






\end{frame}


\begin{frame}
  \frametitle{ Avaluació de la traducció automàtica /2}
{
{Volem avaluar \empha{l'adopció} d'un sistema de traducció
  automàtica per a la \empha{disseminació} (publicació).}
\begin{itemize}
\item Les traduccions en brut s'hauran de \empha{posteditar} (corregir): com
  menys correccions, més \empha{qualitat}: millor.
\item D'acord: com ho avaluem?
\end{itemize}


}
\end{frame}
%%%%%%%%%%%%%%%%%%%%%%%%%%%%%%%%%%%%%%%%%%%%%%%%%%%
%%%%%%%%%%%%%%%%%%%%%%%%%%%%%%%%%%%%%%%%%%%%%%%%%%%%%%%%%%%%%%%%%%%%%
\begin{frame}
  \frametitle{ Avaluació de la traducció automàtica /3}
{
{Per avaluar la \empha{qualitat}, cal:}
\begin{itemize}\itemsep 0ex
{\item elegir una mostra suficient de textos representatius,}
{\item traduir-la automàticament,}
{\item i comptar la quantitat de correcció \empha{mínima} necessària per
  a fer que la traducció siga \empha{adequada al propòsit previst}.}
\end{itemize}
{Sembla senzill, però...}


}
\end{frame}
%%%%%%%%%%%%%%%%%%%%%%%%%%%%%%%%%%%%%%%%%%%%%%%%%%%%%%%%%%%%%%%%%%%%%
\begin{frame}
  \frametitle{ Avaluació de la traducció automàtica /4}
{
{...no ho és gens!}
\begin{itemize}\itemsep 0ex
{\item és difícil elegir prou text representatiu per endavant;}
{\item la noció d'\empha{adequació} és de vegades difícil d'especificar:}
{\item és difícil fer el \empha{mínim} de correccions (cal buscar
  traduccions adequades que se n'obtinguen amb poques correccions);}
{\item tot el procés és molt costós (temps de correcció).}
\end{itemize}

}
\end{frame}
%%%%%%%%%%%%%%%%%%%%%%%%%%%%%%%%%%%%%%%%%%%%%%%%%%%
%%%%%%%%%%%%%%%%%%%%%%%%%%%%%%%%%%%%%%%%%%%%%%%%%%%%%%%%%%%%%%%%%%%%%
\begin{frame}
  \frametitle{ Avaluació de la traducció automàtica /5}
{
{Però la qualitat dels textos traduïts en brut no ho és tot!}

{Fem un \textbf{pressupost}: si adoptem la traducció automàtica,}

{d'una banda, ens estalviem els costos de traducció humana,}

{però tenim despeses noves:}

\begin{itemize}\itemsep 0ex
{\item despeses de \empha{funcionament} i}
{\item despeses de \empha{formació} (s'ha d'aprendre a usar una nova
  tecnologia)} 
\end{itemize}

}
\end{frame}
%%%%%%%%%%%%%%%%%%%%%%%%%%%%%%%%%%%%%%%%%%%%%%%%%%%
%%%%%%%%%%%%%%%%%%%%%%%%%%%%%%%%%%%%%%%%%%%%%%%%%%%%%%%%%%%%%%%%%%%%%
\begin{frame}
  \frametitle{ Avaluació de la traducció automàtica /6}
{
{Despeses de funcionament:}
\begin{itemize}\itemsep 0ex
  {\item \textbf{Cost del sistema de TA} (cost efectiu per mot):
    amortització (sistema en propietat), cost per mot (sistema llogat), servei
    tècnic i manteniment, costos de migració (adaptació de programes,
    adquisició de sistemes), i (no oblidem) el cost d'avaluació!}
  {\item \textbf{Cost de preedició i preparació:} potser cal preparar i 
    \emph{preeditar} els textos i això ho ha de fer algú, cobrant.}   {\item
      \textbf{Cost de postedició}: depén de la \empha{qualitat}; pot baixar amb
      la formació; depén de com paguem als posteditors (per mot, per temps),
      etc.}
\end{itemize}

}
\end{frame}
%%%%%%%%%%%%%%%%%%%%%%%%%%%%%%%%%%%%%%%%%%%%%%%%%%%%%%%%%%%%%%%%%%%%%
%%%%%%%%%%%%%%%%%%%%%%%%%%%%%%%%%%%%%%%%%%%%%%%%%%%%%%%%%%%%%%%%%%%%%
\begin{frame}
  \frametitle{ Avaluació de la traducció automàtica /7}
{
{Despeses de formació:}
\begin{itemize}\itemsep 0ex
  {\item \textbf{Formació en ús del programa de TA}: ús pròpiament dit,
    configuració i manteniment; ús de nou programari associat.}  
  
  {\item \textbf{Formació en postedició:}}
    \begin{itemize}\itemsep 0ex
    {\item coneixement del programa de TA
    (errors típics);}
    {\item tècniques de correcció, ús avançat del processador de textos,
    macroinstruccions, substitució de patrons, etc.}
    \end{itemize}
\end{itemize}
}

\end{frame}
%%%%%%%%%%%%%%%%%%%%%%%%%%%%%%%%%%%%%%%%%%%%%%%%%%%%%%%%%%%%%%%%%%%%%
\begin{frame}
  \frametitle{ Avaluació de la traducció automàtica /8}

{
  \begin{itemize}
  \item {I potser ens hem deixat encara alguna cosa!}
\pause

\item {Avaluar la traducció automàtica no és gens fàcil.}
\pause

\item {La lliçò? Desconfieu de les primeres impressions.}
  \end{itemize}

}
\end{frame}
%%%%%%%%%%%%%%%%%%%%%%%%%%%%%%%%%%%%%%%%%%%%%%%%%%%

\section{TA per al català}
\begin{frame}
  \frametitle{Traducció automàtica per al català/1}
%  \textbf{[un recompte del que hi ha]}
Hi ha bastants sistemes de TA per al català:
\begin{itemize}
\item \textbf{SALT} (v. 4.0), de la Generalitat Valenciana
  (\Pair{es}{ca}): instal·lable en Linux, Windows i Mac i accessible
  en línia (p.ex. \empha{Las Provincias}): gratuït (no lliure), gran
  cobertura lèxica, concebut per aprendre valencià.
  \begin{itemize}
  \item \url{http://www.edu.gva.es/polin/val/salt/apolin_salt.htm}
  \end{itemize}
\item \textbf{interNOSTRUM.com}, CAM/Universitat d'Alacant (\Pair{es}{ca}): en línia, molt usat en Internet, però ja no es manté.
  \begin{itemize}
  \item \url{http://www.internostrum.com}
  \end{itemize}
\end{itemize}
\end{frame}

\begin{frame}
  \frametitle{Traducció automàtica per al català/2}

Més sistemes:
\begin{itemize}
\item \textbf{Apertium.org} (\Pair{es}{ca}, \Pair{en}{ca}, \Pair{fr}{ca}, \Pair{pt}{ca}, \Pair{ca}{oc},  \pair{ca}{eo}): hereu d'\empha{interNOSTRUM.com}, en línia i descarregable, lliure/de codi font obert (l'explicaré més avall).
  \begin{itemize}
  \item \url{http://www.apertium.org}
  \end{itemize}
\item \textbf{Lucy Software} (\Pair{es}{ca}, \Pair{en}{ca}, \Pair{fr}{ca}, \Pair{ca}{de}, \pair{ca}{oc}): comercial, es pot provar en línia (abans \empha{Incyta}, \empha{Comprendium}, \empha{Translendium}\ldots).
  \begin{itemize}
  \item \url{http://www.lucysoftware.com/catala/inici/}
  \end{itemize}
\item \textbf{Automatic Trans} (\Pair{es}{ca}, \Pair{eu}{ca}, \Pair{gl}{ca}, \Pair{pt}{ca}): comercial, grans clients, relacionat amb l'usat cada dia a \empha{El Periódico de Catalunya}.
  \begin{itemize}
  \item \url{http://www.cat.automatictrans.es/}
  \end{itemize}
\end{itemize}

\end{frame}
\begin{frame}
  \frametitle{Traducció automàtica per al català/2}

Més sistemes:
\begin{itemize}
\item \textbf{SisHiTra} (\Pair{es}{ca}), UPV: ``Sistema Híbrid de Traducció'' que combina regles lingüístiques amb models estadístics, resultats molt interessants.
  \begin{itemize}
  \item \url{http://sishitra.iti.upv.es/}
  \end{itemize}
\item \textbf{Google Translate} (de \texttt{ca} a més de 50 llengües): sistema estadístic de la companyia Google.
  \begin{itemize}
  \item \url{http://translate.google.com}
  \end{itemize}
\item \textbf{Bing Translator} (de \texttt{ca} a més de 20 llengües): sistema estadístic de la companyia Microsoft.
  \begin{itemize}
  \item \url{http://www.microsofttranslator.com/}
  \end{itemize}
\end{itemize}

\end{frame}

\begin{frame}
  \frametitle{Traducció automàtica per al català /4}

\textbf{Quin és el millor sistema?}

Consell: abans de prendre una decisió, proveu-los tots, tenint en compte
\begin{itemize}
\item la qualitat de la traducció per al vostre propòsit
\item el preu
\item possibles limitacions a la grandària o el format dels textos
\item si preserven o no el format del text, etc.
\end{itemize}
Recordeu el que hem discutit abans quant a l'avaluació!
\end{frame}


\section[Traducció automàtica lliure]{TA lliure o de codi font obert}


\begin{frame}
\frametitle{Què és el programari lliure o de codi font obert?}
El programari és \empha{lliure} (Free Software Foundation, \url{www.fsf.org}) quan:
\begin{itemize}
\item[0] hom es lliure d'usar el programa per a qualsevol propòsit
\item[1] hom pot estudiar com funciona el programa, i adaptar-lo a les seues necessitats
\item[2] hom pot distribuir còpies i ajudar així el veí
\item[3] hom pot millorar el programa i fer públiques les millores als altres, de manera que tota la comunitat se'n beneficie
\end{itemize}
Perquè les condicions 1 i 3 es complisquen, s'ha de tenir accés al codi font (tal com l'ha escrit el programador). Per això se'n diu també \empha{programari de codi font obert} (Open Source Initiative, \url{www.opensource.org}).
\end{frame}

%\subsection*{Programari de traducció automàtica: obert o tancat?}

\begin{frame}
\frametitle{Peculiaritats del programari de traducció automàtica}
\begin{itemize}
\item La traducció automàtica (TA) és especial: depén fortament de l'existència de dades. Hi ha tres components en qualsevol sistema de TA:\footnote{TA ``basada en regles''; la TA ``basada en corpus'' té requisits anàlegs}
  \begin{itemize} 
  \item El \empha{motor} (el programa pròpiament dit)
  \item Les \empha{dades lingüístiques} (diccionaris, regles)
  \item Les \empha{eines} necessàries per a mantenir aquestes dades i convertir-les al format usat pel \empha{motor}
  \end{itemize}
\end{itemize}


\end{frame}

\begin{frame}
\frametitle{La traducció automàtica comercial, normalment de codi font tancat}
  \begin{itemize}
  \item Els sistemes comercials usen tecnologies \empha{privatives} o
    \empha{de propietat} (\empha{proprietary}) que no es revelen (el fabricant
    les percep com un avantatge competitiu fonamental)
  \item Típicament, només s'hi permet la modificació parcial (\empha{personalització}) de les dades lingüístiques.
  \item Que un sistema es puga usar a Internet no vol dir que siga lliure o de codi font obert (versions de prova, sistemes no comercials).
  \end{itemize}
\end{frame}

\begin{frame}
\frametitle{Traducció automàtica lliure o de codi font obert}
Perquè la TA siga lliure o de codi font obert, tant   
\begin{itemize}
\item el motor, 
\item les dades, 
\item com les ferramentes 
\end{itemize}
han de ser lliures o de codi font obert.
\end{frame}

\begin{frame}
\frametitle{Avantatges de la traducció automàtica lliure o de codi font obert}
\begin{itemize}
\item L'ús de sistemes de TA lliures o de codi font obert té
  avantatges específics sobre els sistemes comercials de codi font
  tancat.  Destaquem-ne dos:
 \begin{itemize}
 \item \empha{Augmenta la perícia} (\empha{expertise}, coneixement) de
   les comunitats lingüístiques implicades i \empha{els recursos
     lingüístics} disponibles per a elles (fixació, codificació i
   aplicació del coneixement), i els dissemina molt eficientment.
 \item \empha{Augmenta la independència} respecte d'un proveïdor
   comercial de traducció automàtica tancada (o d'altres tecnologies
   lingüístiques).
 \end{itemize}
\end{itemize}
\end{frame}



\begin{frame}
  \frametitle{Reptes de la traducció automàtica lliure o de codi font
    obert} Per a poder gaudir d'aquests avantatges, les comunitats
  lingüístiques implicades en la creació d'un nou sistema de TA han de
  fer front a reptes com :
  \begin{itemize}
  \item La neutralització d'actituds \empha{tecnofòbiques} (!)
  \item L'organització del desenvolupament comunitari
  \item L'\empha{elicitació} del coneixement lingüístic
  \item L'estandardització i documentació dels formats de les dades lingüístiques
  \item L'assoliment de la modularitat en  programes i dades.
  \end{itemize}

\end{frame}




\section{Apertium: un sistema de TA lliure/de codi font obert}

%\subsection*{Antecedents i fonaments}
\begin{frame}
 \frametitle{Antecedents} 

 Apertium està basat en les tecnologies creades pel grup Transducens
 de la Universitat d'Alacant durant el desenvolupament de dos sistemes
 existents:
 
\begin{itemize}
 
\item \textbf{interNOSTRUM} (\texttt{interNOSTRUM.com},
  \texttt{es}\(\leftrightarrow\)\texttt{ca}%\footnote{\scriptsize{La llengua amb codi ISO-639 \texttt{ca}, parlada a Xàtiva, Palma, Olot, Perpinyà\ldots, és coneguda com a \empha{català}, \empha{valencià}, \empha{mallorquí}, etc. i la immensa majoria dels qui l'escriuen bé usen bàsicament el mateix estàndard.}})
 
\item \textbf{Tradutor Universia} (\texttt{tradutor.universia.net},
  \texttt{es}\(\leftrightarrow\)\texttt{pt})
 
\end{itemize}

Aquestes tecnologies, inicialment dissenyades per a parells de
llengües estretament emparentades (com les romàniques!), han estat
esteses per a tractar parells de llengües més allunyats.

\end{frame}

%\subsection*{Fonaments}

\begin{frame}
\frametitle{Fonaments /1}

Per a generar traduccions que siguen: 

\begin{itemize}

\item raonablement intel·ligibles i
\item fàcils de corregir (posteditar)
\end{itemize}
entre llengües estretament emparentades com l'espanyol (\texttt{es}) i
el català (\texttt{ca}) o el portugués (\texttt{pt}), etc., només cal
millorar la traducció \empha{mot per mot} amb algunes operacions.

\end{frame}

\begin{frame}
\frametitle{Fonaments /2}

Millores sobre la traducció mot per mot: 
\begin{itemize}
\item processament lèxic robust (incloent-hi unitats lèxiques multi-mot)
  \begin{center}
    (es) \textit{echar de menos} \(\to\) (ca) \sout{\textit{tirar de menys}} \textit{trobar a faltar}
  \end{center}
\item desambiguació lèxica categorial (\empha{part-of-speech tagging})
  \begin{center}
    (es) \textit{amigas como ella} \(\to\) (ca) \textit{amigues \sout{menge} \textbf{com} ella} 
  \end{center}
\item processament estructural local basat en regles simples i ben formulades per a transformacions estructurals freqüents (reordenació, concordança):
  \begin{center}
    (es) \textit{un valle seco} \(\to\) (ca) \textit{\sout{un} \textbf{una} vall \sout{sec} \textbf{seca}}
  \end{center}
\end{itemize}

\end{frame}

\begin{frame}
  \frametitle{Fonaments /3}
Per a parells de llengües més difícils, no tan 
relacionats: 
\begin{itemize}
\item Hauria de ser possible estendre aquest model senzill.
\item Hauria de ser possible generalitzar-ne els conceptes de manera que la complexitat es mantinga tan baixa com siga possible.
\end{itemize}
\end{frame}


\begin{frame}
  \frametitle{Fonaments /4}
  
\begin{itemize}
  
\item Hauria de ser possible generar un sistema complet de traducció
  automàtica a partir de dades lingüístiques (diccionaris monolingües
  i bilingües, regles gramaticals), especificades de manera
  \textbf{declarativa}.
\item Aquestes dades: 
\begin{itemize}
\item regles (independents de la llengua) per a tractar formats de text
\item especificació del desambiguador lèxic categorial
\item diccionaris morfològics i bilingües i diccionaris de regles de transformació ortogràfica
\item regles de transferència estructural
\end{itemize}
haurien d'estar en un format interoperable  $\Rightarrow$ \textbf{XML}.
  
\end{itemize}

\end{frame}
\begin{frame}
  \frametitle{Fonaments /5}
  
\begin{itemize}
  
\item Hauria de ser possible tenir un motor de traducció únic (independent de la llengua) que llegiria dades específiques per a cada parell de llengües
  (``separació d'algorismes i dades'').
\item Les dades lingüístiques del parell de llengües haurien de ser preprocessades de manera que el sistema siga ràpid ($>$10,000 mots per segon) i compacte; per exemple, les transformacions lèxiques es farien amb transductors d'estats finits
  (TEFs).
\end{itemize}

\end{frame}
\begin{frame}
  \frametitle{Fonaments /6}
  
  Raons per al desenvolupament d'Apertium com a programari lliure o de codi font obert:
\begin{itemize}
\item Donar a tothom accés lliure i il·limitat a les millors tecnologies possibles de traducció automàtica.
\item Establir una plataforma modular, documentada i oberta per a la traducció automàtica de transferència superficial i per a altres tasques de processament automàtic de la llengua.
\item Afavorir l'intercanvi i la reutilització de les dades lingüístiques existents.
\item Facilitar la integració amb altres tecnologies lliures o de codi font obert.
\end{itemize}

\end{frame}
\begin{frame}
  \frametitle{Fonaments /7}
  
  Més raons per al desenvolupament d'Apertium com a programari lliure
  o de codi font obert:

\begin{itemize}
\item Beneficiar-se del desenvolupament col·laboratiu 
  \begin{itemize}
  \item del motor de traducció i de les eines
  \item de dades per a parells de llengües existents o nous
  \end{itemize}
per part de la indústria, de les universitats o d'organitzacions de suport de llengües menors.
\item Promoure el canvi de model de negoci en traducció automàtica,
  de \empha{basat en llicències} a \empha{basat en serveis}.
\item Garantir radicalment la reproducibilitat de la recerca en TA.
\item Perquè no té sentit usar diners públics per a desenvolupar
  programari no lliure i de codi font tancat.
\end{itemize}

\end{frame}

%\subsection*{La plataforma Apertium}
\begin{frame}
  \frametitle{La plataforma Apertium} 
Apertium és una plataforma de traducció automàtica de codi obert (\url{http://www.apertium.org}) que proporciona:
\begin{enumerate}
\item Un \textbf{motor} de traducció automàtica, basat en transferència sintàctica superficial, amb:
  \begin{itemize}
  \item gestió de formats de text
  \item processament lèxic basat en estats finits
  \item transferència sintàctica superficial basada en reconeixement de patrons basat en estats finits
  \end{itemize}
\item \textbf{Dades lingüístiques} en formats XML ben especificats per un nombre creixent de parells de llengües
\item \textbf{Compiladors} per a passar les dades a la forma compacta i ràpida usada pel motor i programari per a aprendre regles de desambiguació o de transferència estructural.
\end{enumerate}
\end{frame}

%subsection*{El motor d'Apertium}
\begin{frame}
  \frametitle{El motor d'Apertium/1}
{\scriptsize
  \begin{tabular}{cccccc}
    \textbf{Text SL}$\to$ & \framebox{Desformatador} & & \\
     & $\downarrow$        & & \\
     & \framebox{Analitzador morfològic} & & [$\gets$TEF] \\
     & $\downarrow$        & & \\
     & \framebox{Desambiguador categorial} & & [$\gets$TEF+estad.] \\
     & $\downarrow$        & & \\
          {}[regles$\to$]      & \framebox{Transferència estructural} & $\leftrightarrow$ \framebox{Transferència lèxica} &[$\gets$TEF] \\
     & $\downarrow$        & & \\ 
     & \framebox{Generador morfològic} & & [$\gets$TEF]   \\
     & $\downarrow$        & & \\
      & \framebox{Post-generador} & & [$\gets$TEF] \\
     & $\downarrow$        & & \\
      & \framebox{Reformatador} & $\to$\textbf{text LM} \\
  \end{tabular}
}

\end{frame}


\begin{frame}
\frametitle{El motor d'Apertium/2}
  Comunicació entre els mòduls: \textbf{text} legible (\empha{canonades} o \empha{pipelines} d'Unix).

Avantatges:
\begin{itemize}
  
\item Simplifica la diagnosi i la depuració d'errors (examinant
  l'eixida de cada mòdul)
\item Permet la modificació de dades entre dos mòduls, usant, per
  exemple, programes més menuts (anomenats en Unix \empha{filtres})
\item Facilita la inserció de mòduls alternatius (interessant per a la recerca i el desenvolupament)
\end{itemize}

\end{frame}

%\subsection*{Dades lingüístiques}

\begin{frame}
\frametitle{Dades lingüístiques}
\begin{itemize}
\item Apertium acull el desenvolupament de dades per a un gran nombre de parells de llengües,
\item  amb \empha{èmfasi especial} sobre les \empha{llengües romàniques}, 
\end{itemize}
com veurem dins d'un moment.
\end{frame}


%\subsection*{Finançament}

\begin{frame}
\frametitle{Finançament}
Finançat per: 
\begin{itemize}
\item Ministeris d'Indústria, Turisme i Comerç, d'Educació i Ciència i  de Ciència i Tecnologia d'Espanya.
\item Secretaria de Telecomunicacions i Societat de la Informació de la Generalitat de Catalunya
\item El Ministeri d'Assumptes Exteriors de Romania
\item La Univ.\ d'Alacant i la Univ.\ Oberta de Catalunya
\item La \textbf{Ofis ar Brezhoneg} (Oficina del Bretó)
\item Beques del \empha{Google Summer of Code} (2009--2011, 9/any).
\item Empreses: Prompsit Language Engineering, imaxin|software, Eleka
  Ingeniaritza Linguistikoa, ABC Enciklopedioj, Eolaistriu   Technologies, etc.
\item La Direcció General de Traducció de la Commissió Europea
\end{itemize} 
\end{frame}


%\subsection*{Recerca i negocis amb Apertium}
\begin{frame}
  \frametitle{Recerca i negocis amb Apertium}
  Apertium és ja una activa plataforma de recerca i de negocis:
  \begin{itemize}
  \item \textbf{Recerca:} >30 publicacions, 1 tesi doctoral, 4 tesis de màster
  \item \textbf{Negocis:} empreses (Prompsit, Eleka, Imaxin|software, etc.) comercialitzen serveis a clients com ara Autodesk, la Generalitat de Catalunya, un dels bancs més importants d'Euskal Herria, el diari \empha{La Voz de Galicia}, etc.
  \end{itemize}
El model de desenvolupament lliure / de codi font obert crea una \textbf{comunitat} que connecta eficientment els \textbf{investigadors}, els \textbf{desenvolupadors}, els \textbf{comercialitzadors} i els \textbf{usuaris}.
\end{frame}

%\subsection*{La comunitat d'Apertium}
\begin{frame}
\frametitle{La comunitat d'Apertium/1}
A més dels desenvolupadors originals (finançats), s'ha format una comunitat al voltant del projecte (instigada fonamentalment per Francis Tyers).
\begin{itemize}
\item Hi ha >100 desenvolupadors inscrits en
  \texttt{sourceforge.net/projects/apertium/}, la majoria de fora del grup
  original; el codi s'actualitza molt freqüentment (centenars
  d'actualitzacions cada mes).
\item Un \empha{wiki} mantingut col·lectivament documenta els components d'Apertium, mostra l'estat actual de desenvolupament i dóna consells per als desenvolupadors de dades lingüístiques o de programes.
\end{itemize}
\end{frame}
\begin{frame}
\frametitle{La comunitat d'Apertium/2}
\begin{itemize}

\item Exemples d'eines i codi desenvolupat externament: 
  \begin{itemize}
  \item la interfície gràfica d'ús \texttt{apertium-tolk}, i l'eina de diagnòstic \texttt{apertium-view}
   \item \empha{plugins} per a OpenOffice.org, per al missatger Pidgin (abans Gaim), per al gestor de continguts Wordpress, etc.
   \item Una versió dels diccionaris bilingües per a mòbils i PDA (\texttt{tinylex})
   \item Una aplicació per a subtítols (\texttt{apertium-subtitles})
   \item Versions preliminars per a Windows
  \end{itemize}
\item Molts desenvolupadors es troben en el canal IRC \texttt{\#apertium} (de \texttt{freenode.net}).
\item Els paquets estables estan disponibles en Debian GNU/Linux (i per tant, en Ubuntu Linux).
\end{itemize}
\end{frame}


\begin{frame}
  \frametitle{Les llengües romàniques en Apertium}
La majoria dels parells d'Apertium són amb llengües \textbf{romàniques} (i algunes ben menudes!)
  \begin{tabular}{|lll|lll|}
\hline
\textbf{Parell}&\textbf{Versió}&\textbf{Data}&\textbf{Parell}&\textbf{Versió}&\textbf{Data}\\
\hline
    \pair{br}{\textbf{fr}}&0.4.0&07/02/2011&
    \pair{\textbf{ca}}{eo}&0.9.1&04/07/2009\\
    \Pair{en}{\textbf{ca}}&0.9.1&17/09/2010&
    \Pair{en}{\textbf{es}}&0.7.1&23/02/2010\\
    \Pair{en}{\textbf{gl}}&0.5.1&19/11/2008&
    \pair{eu}{\textbf{es}}&0.3.1&24/04/2009\\
    \Pair{\textbf{es}}{\textbf{an}}&0.1.0&26/09/2010&
    \pair{\textbf{es}}{\textbf{ast}}&1.1.0&22/12/2010\\
    \Pair{\textbf{es}}{\textbf{ca}}&1.2.0&15/10/2009&
    \pair{\textbf{es}}{eo}&1.0&17/11/2009\\
    \Pair{\textbf{es}}{\textbf{gl}}&1.0&04/10/2007&
    \Pair{\textbf{es}}{\textbf{pt}}&1.1.0&10/06/2011\\
    \Pair{\textbf{fr}}{\textbf{ca}}&1.0.2&12/03/2009&
    \pair{\textbf{fr}}{eo}&0.9.0&14/02/2011\\
    \Pair{\textbf{fr}}{\textbf{es}}&0.9.0&02/03/2009&
    \pair{\textbf{it}}{\textbf{ca}}&0.1.0&13/01/2010\\
    \Pair{\textbf{oc}}{\textbf{ca}}&1.0.5&21/07/2008&
    \Pair{\textbf{oc}}{\textbf{es}}&1.0.5&21/07/2008\\
    \Pair{\textbf{pt}}{\textbf{ca}}&0.8.1&04/07/2009&
    \Pair{\textbf{pt}}{\textbf{gl}}&0.9.1&03/07/2009\\
    \pair{\textbf{ro}}{\textbf{es}}&0.7&08/10/2007 \\
\hline
  \end{tabular}



\end{frame}




% \begin{frame}
%   \frametitle{Apertium: un sistema de TA lliure/de codi font obert}
%   \textbf{[Apertium en dues o tres transparències: què hi ha dins d'Apertium]}
% \end{frame}

\section{Memòries de traducció (bonus)}
\begin{frame}
\frametitle{Memòries de traducció/1}

{
{Els traductors (humans) han generat 
\empha{moltíssimes traduccions}.}

{Hi ha a l'abast \empha{nombrosos textos electrònics bilingües}
  on la versió en un idioma  és una bona traducció de la versió en l'altre i
  viceversa.}

{No es podria \empha{aprofitar} aquest treball per 
a traduir documents nous (\empha{reciclatge automàtic de 
traduccions}?) $\to$ Alternativa a la traducció automàtica.}
}
\end{frame}

\begin{frame}
\frametitle{ Memòries de traducció /2}

{

{Per a aprofitar aquests \empha{bitextos} cal:}
\begin{itemize}\setlength{\itemsep}{0pt}
{\item \empha{Alinear-los} (indicar quines parts són 
traducció de quines);}
{\item \empha{Segmentar-los} en unitats de traducció (UT);}
{\item \empha{Organitzar les UT} en una base de dades eficient.}
\end{itemize}
{Totes aquestes tasques, tan automàticament com 
siga possible.}
}
\end{frame}

\begin{frame}

\frametitle{ Memòries de traducció/3}

Esquema del procés de \empha{segmentació} i d'\empha{alineament} d'un
  parell de textos existent per a alimentar
  una memòria de traducció.


\small{
$$
\begin{array}{rcl}
\mbox{\parbox{1.0cm}{\textsf{text esquerre $E$}}} &\to \mbox{\framebox{\parbox{2.0cm}{\textsf{segmentació}}}} 
\to \\
\mbox{\parbox{1.0cm}{\textsf{text dret $D$}}} &\to \mbox{\framebox{\parbox{2.0cm}{\textsf{segmentació}}}} 
\to \\
\end{array}
%\right.
\mbox{\framebox{\parbox{1.7cm}{\textsf{alineador- corrector assistit}}}}
\to \mbox{\parbox{1.2cm}{\textsf{UT} $(e_1,d_1)$, $(e_2,d_2)$, \ldots}} \to \mbox{\framebox{\parbox{1.4cm}{\textsf{Memòria de traducció}}}}
$$
}



\end{frame}

\begin{frame}

\frametitle{ Memòries de traducció/4}
{
{Per a traduir textos nous cal:}
\begin{itemize}\setlength{\itemsep}{0pt}
{\item \empha{Segmentar-los} en unitats que puguen 
correspondre amb les UT existents}
{\item \empha{Substituir} els segments trobats per les 
traduccions corresponents.}
\end{itemize}
{Aquest és el fonament de les \empha{memòries de 
traducció}.}

}

\end{frame}

\begin{frame}

\frametitle{ Memòries de traducció/5}

\textbf{Modalitat de pretraducció:}

Esquema del procés de \empha{pretraducció} d'un nou text esquerre $E'$
usant una memòria de traducció.

$$
\begin{array}{rcl}
\mbox{\parbox{1.3cm}{\textsf{text esquerre $E'$}}} \to
\mbox{\framebox{\textsf{segmentació}}} \to &
\mbox{\framebox{\textsf{pretraducció}}} & \to \mbox{\parbox{2.5cm}{\textsf{text dret
    pretraduït i segmentat (per a editar)}}} \\
& \uparrow \downarrow \mbox{\textsf{UT}}& \\[1cm]
& \mbox{\framebox{\parbox{2.2cm}{\textsf{Memòria de traducció}}}} & \\
\end{array}
$$
\end{frame}
\begin{frame}

\frametitle{ Memòries de traducció/5}

\textbf{Modalitat interactiva:}
%\textbf{[completar]}

Esquema del procés de \empha{traducció interactiva} d'un nou text esquerre $E'$
usant una memòria de traducció.

$$
\begin{array}{rcl}
\mbox{\parbox{1.3cm}{\textsf{text esquerre $E'$}}} \to
\mbox{\framebox{\textsf{segmentació}}} \to &
\mbox{\framebox{\textsf{elecció (i edició)}}} & \to \mbox{\parbox{2.5cm}{\textsf{text dret \(D'\)
    completat i segmentat}}} \\
& \uparrow \downarrow \mbox{\textsf{UT}}& \\[1cm]
& \mbox{\framebox{\parbox{2.2cm}{\textsf{Memòria de traducció}}}} & \\
\end{array}
$$
\end{frame}

\begin{frame}

\frametitle{ Memòries de traducció/7}
{
{Alguns productes comercials (preus de 600 euros cap amunt):}
\begin{itemize}\setlength{\itemsep}{0pt}
{\item \empha{Déjà Vu} d'Atril (\texttt{http://www.atril.com}) }
{\item \empha{Transit} de Star
  (\texttt{http://www.star-group.net})}
{\item \empha{SDL Trados} (\texttt{www.trados.com})}
\end{itemize}
{Solen contenir, a més de la \empha{memòria de traducció}, altres
útils com ara \empha{bases de dades lèxiques} (``terminològiques''), etc.}

Hi ha productes \empha{lliures o de codi font obert} com ara 
  \begin{itemize}
  \item \texttt{OmegaT} (\url{www.omegat.org})
  \item \texttt{OpenTM2} (\url{www.opentm2.org})
  \end{itemize}
}

\end{frame}


\begin{frame}

\frametitle{ Memòries de traducció/7}
{
{\empha{Quan funcionen bé} les memòries de 
traducció?}
\begin{itemize}\setlength{\itemsep}{0pt}
{\item Quan tenim moltes traduccions alineades en la memòria}
{\item Quan els tipus de textos a traduir són 
\empha{molt repetitius}}
{\item Quan la \empha{terminologia} i la \empha{fraseologia} són 
\empha{estables} en la memòria}
\end{itemize}
{Però:}
\begin{itemize}\setlength{\itemsep}{0pt}
{\item  sempre cal revisar la pretraducció}
{\item A canvi: la pretraducció revisada es pot 
afegir ja a la memòria de traducció per usar-la en el futur.}
\end{itemize}
}
\end{frame}


\begin{frame}
\frametitle{ Memòries de traducció/8}
{
{Sobre la segmentació:}
\begin{itemize}\setlength{\itemsep}{0pt}
{\item Els programes de MT segmenten els textos en 
\empha{``oracions''} usant la \empha{puntuació} i el \empha{format}.}
{\item A canvi, troben en la memòria segments \empha{aproximats} a 
més dels idèntics (i produeixen traduccions aproximades).}
{\item Hi ha (des de 1998) un format estàndard internacional
de MT independent del programa: TMX 
(Translation Memory eXchange), que permet l'intercanvi de memòries
entre equips de traducció.}
\end{itemize}
}
\end{frame}

\begin{frame}

  © 2011 Universitat d'Alacant

  © 2011 Mikel L. Forcada

Aquest treball es pot distribuir lliurement en els termes de qualsevol d'aquestes dues llicències:
\begin{itemize}
\item la llicència Creative Commons
  Attribution--Share Alike:
  \url{http://creativecommons.org/licenses/by-sa/3.0/deed.ca} 
\item la llicència GNU GPL v. 3.0: \url{http://www.gnu.org/licenses/gpl.html}
\end{itemize}
Per a obtenir els fonts \LaTeX{}, només cal escriure a:\texttt{mlf@ua.es}


\end{frame}


\end{document}
